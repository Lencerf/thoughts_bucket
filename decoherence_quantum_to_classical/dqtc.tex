\documentclass{article}
\title{Notes for \textsl{Decoherence and the
Quantum-to-Classical Transition} by Dr. Maximilian Schlosshauer}
\author{L}
\newcommand{\separator}{\par\rule{\textwidth}{0.4pt}\vskip 0.2cm}
\begin{document}
\maketitle
\section{Introducing Decoherence}

\section{The Basic Formalism and Interpretation of Decoherence}

The measurement problem(the problem of the quantum to classical transition):
\begin{itemize}
    \item The problem of the preferred basis: why are physical systems usually 
    observed to be in definite positions(eigenstates of position) rather than
    in superpositions of positions(basis of other operators)?
    \item The problem of the nonobservability of interference: why is it 
    difficult to observe quantum interference effects, especially on macroscopic
    scales?
    \item The problem of outcomes: what selects a particular outcome among the
    different possible outcomes described by the quantum probability distribution?
\end{itemize}
Decoherence answers the first two questions. The third problem is a fundamental
issue of quantum mechanics.

\end{document}