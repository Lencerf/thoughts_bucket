\documentclass{article}
\title{Notes for \textsl{Decoherence and the
Quantum-to-Classical Transition} by Dr. Maximilian Schlosshauer}
\author{L}
\newcommand{\separator}{\par\rule{\textwidth}{0.4pt}\vskip 0.2cm}
\setlength\parindent{0pt}
\setlength{\parskip}{0.5\baselineskip}
\usepackage{enumitem}
\setlist[itemize]{noitemsep, topsep=0pt}
\begin{document}
\maketitle
\section{Introducing Decoherence}

\section{The Basic Formalism and Interpretation of Decoherence}
2.1.1\\ The identification between formal operators and the physical quantities is 
indtoduced axiomaticaly nto the quantum theory and is therefore \textbf{nontrival}.

2.1.2\\ Quantum states represent a complete description of a quantum system, although
in general quantum states do not tell us which particular outcome will be obtained
in a measurement but only the probabilities of possible outcomes.

Local hidden-variables theories are ruled out by violations of Bell's inequality
in experiments. Any hidden-variables theory must be contextual: the hidden variable
ascribed to a quuantum system will be dependent on the measurement context. But 
acutally we want a hidden-variable theory so that the physical world is independent
of any measurements performed on it.

The measurement problem(the problem of the quantum to classical transition):
\begin{itemize}
    \item The problem of the preferred basis: why are physical systems usually 
    observed to be in definite positions(eigenstates of position) rather than
    in superpositions of positions(basis of other operators)?
    \item The problem of the nonobservability of interference: why is it 
    difficult to observe quantum interference effects, especially on macroscopic
    scales?
    \item The problem of outcomes: what selects a particular outcome among the
    different possible outcomes described by the quantum probability distribution?
\end{itemize}
Decoherence answers the first two questions. The third problem is a fundamental
issue of quantum mechanics.

\end{document}